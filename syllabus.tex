% Options for packages loaded elsewhere
% Options for packages loaded elsewhere
\PassOptionsToPackage{unicode}{hyperref}
\PassOptionsToPackage{hyphens}{url}
\PassOptionsToPackage{dvipsnames,svgnames,x11names}{xcolor}
%
\documentclass[
  english,
  letterpaper,
  DIV=11,
  numbers=noendperiod]{scrartcl}
\usepackage{xcolor}
\usepackage{amsmath,amssymb}
\setcounter{secnumdepth}{-\maxdimen} % remove section numbering
\usepackage{iftex}
\ifPDFTeX
  \usepackage[T1]{fontenc}
  \usepackage[utf8]{inputenc}
  \usepackage{textcomp} % provide euro and other symbols
\else % if luatex or xetex
  \usepackage{unicode-math} % this also loads fontspec
  \defaultfontfeatures{Scale=MatchLowercase}
  \defaultfontfeatures[\rmfamily]{Ligatures=TeX,Scale=1}
\fi
\usepackage{lmodern}
\ifPDFTeX\else
  % xetex/luatex font selection
\fi
% Use upquote if available, for straight quotes in verbatim environments
\IfFileExists{upquote.sty}{\usepackage{upquote}}{}
\IfFileExists{microtype.sty}{% use microtype if available
  \usepackage[]{microtype}
  \UseMicrotypeSet[protrusion]{basicmath} % disable protrusion for tt fonts
}{}
\makeatletter
\@ifundefined{KOMAClassName}{% if non-KOMA class
  \IfFileExists{parskip.sty}{%
    \usepackage{parskip}
  }{% else
    \setlength{\parindent}{0pt}
    \setlength{\parskip}{6pt plus 2pt minus 1pt}}
}{% if KOMA class
  \KOMAoptions{parskip=half}}
\makeatother
% Make \paragraph and \subparagraph free-standing
\makeatletter
\ifx\paragraph\undefined\else
  \let\oldparagraph\paragraph
  \renewcommand{\paragraph}{
    \@ifstar
      \xxxParagraphStar
      \xxxParagraphNoStar
  }
  \newcommand{\xxxParagraphStar}[1]{\oldparagraph*{#1}\mbox{}}
  \newcommand{\xxxParagraphNoStar}[1]{\oldparagraph{#1}\mbox{}}
\fi
\ifx\subparagraph\undefined\else
  \let\oldsubparagraph\subparagraph
  \renewcommand{\subparagraph}{
    \@ifstar
      \xxxSubParagraphStar
      \xxxSubParagraphNoStar
  }
  \newcommand{\xxxSubParagraphStar}[1]{\oldsubparagraph*{#1}\mbox{}}
  \newcommand{\xxxSubParagraphNoStar}[1]{\oldsubparagraph{#1}\mbox{}}
\fi
\makeatother


\usepackage{longtable,booktabs,array}
\usepackage{calc} % for calculating minipage widths
% Correct order of tables after \paragraph or \subparagraph
\usepackage{etoolbox}
\makeatletter
\patchcmd\longtable{\par}{\if@noskipsec\mbox{}\fi\par}{}{}
\makeatother
% Allow footnotes in longtable head/foot
\IfFileExists{footnotehyper.sty}{\usepackage{footnotehyper}}{\usepackage{footnote}}
\makesavenoteenv{longtable}
\usepackage{graphicx}
\makeatletter
\newsavebox\pandoc@box
\newcommand*\pandocbounded[1]{% scales image to fit in text height/width
  \sbox\pandoc@box{#1}%
  \Gscale@div\@tempa{\textheight}{\dimexpr\ht\pandoc@box+\dp\pandoc@box\relax}%
  \Gscale@div\@tempb{\linewidth}{\wd\pandoc@box}%
  \ifdim\@tempb\p@<\@tempa\p@\let\@tempa\@tempb\fi% select the smaller of both
  \ifdim\@tempa\p@<\p@\scalebox{\@tempa}{\usebox\pandoc@box}%
  \else\usebox{\pandoc@box}%
  \fi%
}
% Set default figure placement to htbp
\def\fps@figure{htbp}
\makeatother



\ifLuaTeX
\usepackage[bidi=basic]{babel}
\else
\usepackage[bidi=default]{babel}
\fi
% get rid of language-specific shorthands (see #6817):
\let\LanguageShortHands\languageshorthands
\def\languageshorthands#1{}
\ifLuaTeX
  \usepackage[english]{selnolig} % disable illegal ligatures
\fi


\setlength{\emergencystretch}{3em} % prevent overfull lines

\providecommand{\tightlist}{%
  \setlength{\itemsep}{0pt}\setlength{\parskip}{0pt}}



 


\KOMAoption{captions}{tableheading}
\makeatletter
\@ifpackageloaded{caption}{}{\usepackage{caption}}
\AtBeginDocument{%
\ifdefined\contentsname
  \renewcommand*\contentsname{Table of contents}
\else
  \newcommand\contentsname{Table of contents}
\fi
\ifdefined\listfigurename
  \renewcommand*\listfigurename{List of Figures}
\else
  \newcommand\listfigurename{List of Figures}
\fi
\ifdefined\listtablename
  \renewcommand*\listtablename{List of Tables}
\else
  \newcommand\listtablename{List of Tables}
\fi
\ifdefined\figurename
  \renewcommand*\figurename{Figure}
\else
  \newcommand\figurename{Figure}
\fi
\ifdefined\tablename
  \renewcommand*\tablename{Table}
\else
  \newcommand\tablename{Table}
\fi
}
\@ifpackageloaded{float}{}{\usepackage{float}}
\floatstyle{ruled}
\@ifundefined{c@chapter}{\newfloat{codelisting}{h}{lop}}{\newfloat{codelisting}{h}{lop}[chapter]}
\floatname{codelisting}{Listing}
\newcommand*\listoflistings{\listof{codelisting}{List of Listings}}
\makeatother
\makeatletter
\makeatother
\makeatletter
\@ifpackageloaded{caption}{}{\usepackage{caption}}
\@ifpackageloaded{subcaption}{}{\usepackage{subcaption}}
\makeatother
\usepackage{bookmark}
\IfFileExists{xurl.sty}{\usepackage{xurl}}{} % add URL line breaks if available
\urlstyle{same}
\hypersetup{
  pdftitle={Syllabus BIOS 6312},
  pdfauthor={Chris Slaughter},
  pdflang={en},
  colorlinks=true,
  linkcolor={blue},
  filecolor={Maroon},
  citecolor={Blue},
  urlcolor={Blue},
  pdfcreator={LaTeX via pandoc}}


\title{Syllabus BIOS 6312}
\author{Chris Slaughter}
\date{2026-01-02}
\begin{document}
\maketitle


\subsection{Course Goals}\label{course-goals}

\begin{itemize}
\tightlist
\item
  Provide students with a solid foundation in modern regression methods
  used in biostatistical analyses
\item
  Learn how to use modern regression methods to answer scientific
  questions
\item
  Become familiar with statistical concepts including exploratory data
  analysis, estimation, testing in linear, logistic, and survival models
\item
  Understand how the development of statistical methodology is motivated
  by biological and medical problems
\item
  Develop data analytic skills including familiarity with several
  statistical programs
\item
  Develop writing skills needed to communicate the results of a data
  analysis
\item
  Introduce reproducible research approaches using R, Rstudio, and
  Quarto
\end{itemize}

\subsection{Topics of Discussion}\label{topics-of-discussion}

\begin{itemize}
\tightlist
\item
  Bayesian and Frequentist approaches to fitting regression models
\item
  Linear regression
\item
  Logistic regression
\item
  Poisson regression
\item
  Survival models, primarily Cox Regression
\item
  Multinomial and ordinal logistic regression
\item
  Multivariable Regression
\item
  Matrix algebra and important results of random vectors
\item
  Precision, effect modification, and confounding
\item
  Specification issues in regression models
\item
  Model selection
\item
  Case Studies
\item
  Understanding model assumptions and the impact of assumptions on
  interpretation
\item
  Model Checking: diagnostics, transformations, influential
  observations, lack-of-fit test
\end{itemize}

\subsection{Course Description}\label{course-description}

\subsubsection{BIOS 6312}\label{bios-6312}

This is the second in a two-course series designed for students who seek
to develop skills in modern biostatistical reasoning and data analysis.
Students learn modern regression analysis and modeling building
techniques from an applied perspective. Theoretical principles will be
demonstrated with real-world examples from biomedical studies. This
course requires substantial statistical computing in software packages
and focuses on R; familiarity with R or proficiency in another
high-level statistical program (e.g.~Stata) is required. The course
covers regression modeling for continuous outcomes, including simple
linear regression, multiple linear regression, and analysis of variance
with one-way, two-way, three-way, and analysis of covariance models.
This is a brief introduction to models for binary outcomes (logistic
models), ordinal outcomes (proportional odds models), count outcomes
(Poisson/negative binomial models), and time to event outcomes
(Kaplan-Meier curves, Cox proportional hazard modeling). Incorporated
into the presentation of these models are subtopic topics such as
regression diagnostics, nonparametric regression, splines, data
reduction techniques, model validation, parametric bootstrapping, and a
brief introduction to methods for handling missing data. Students are
required to take 6312L concurrently. Prerequisite: Biostatistics 6311 or
equivalent; familiarity with R or Stata software packages. SPRING.

\subsubsection{BIOS 6312L}\label{bios-6312l}

This is a discussion section/lab for Modern Regression Analysis.
Students will review relevant theory and work on applications as a
group. Computing solutions and extensions will be emphasized. Students
are required to take 6312 concurrently.

\subsection{Course Materials}\label{course-materials}

\subsubsection{Course notes}\label{course-notes}

\begin{itemize}
\tightlist
\item
  Course notes will be the primary source
\item
  Available on web page
\item
  Daily class schedule will indicate notes being covered
\item
  Notes will be updated throughout semester
\end{itemize}

\subsubsection{Textbooks}\label{textbooks}

\begin{itemize}
\item
  There are no required textbooks for this course
\item
  The following are provided as references that are at an appropriate
  level for this course

  \begin{itemize}
  \item
    Regression Methods in Biostatistics. Vittinghoff, Glidden, Shiboski,
    and McCulloch
  \item
    Applied Liner Regression. Weisberg.
  \item
    Bayesian and Frequentist Regression Methods. Wakefield
  \end{itemize}
\end{itemize}

\subsection{Grading and Evaluation}\label{grading-and-evaluation}

\subsubsection{Evaluation components and grade
percentages}\label{evaluation-components-and-grade-percentages}

\begin{itemize}
\tightlist
\item
  Midterm (25\%)
\item
  Take Home Exam (25\%)
\item
  Final Exam (25\%)
\item
  Homework (25\%)
\item
  Class participation
\item
  This is a 4-credit course. Your lab and lecture grades will be the
  same
\end{itemize}

\subsubsection{Homework}\label{homework}

\begin{itemize}
\tightlist
\item
  Up to 1 per week (probably 6 or 7 total)
\item
  Will focus on real data analysis and interpretation with some
  mathematical derivations of important quantities
\item
  Questions will focus on specific analyses, with questions stated in as
  scientific terms as possible
\item
  Work handed in should address the scientific questions

  \begin{itemize}
  \tightlist
  \item
    Format Table and Figures
  \end{itemize}
\item
  Keys will be provided shortly after the homework is turned in

  \begin{itemize}
  \tightlist
  \item
    No late homework accepted after the key is posted
  \end{itemize}
\item
  Answers in keys may go beyond what is expected of your homework and
  present concepts in more detail. You are responsible for any material
  in the keys for exams.
\item
  You may discuss the homework with others in the class, but the work
  you turn in should be your own
\item
  Use Brightspace to turn in homeworks and receive feedback and grade
\end{itemize}

\subsubsection{In Class Exams}\label{in-class-exams}

\begin{itemize}
\tightlist
\item
  Midterm and Final in class

  \begin{itemize}
  \tightlist
  \item
    Focus on understanding concepts, not memorizing formulas
  \item
    I will provide an example midterm and final
  \item
    For midterm, you will be allowed 1 page of your own notes
  \item
    For final, you will be allowed 2 pages of your own notes
  \end{itemize}
\item
  All output will be provide for you to interpret
\end{itemize}

\subsubsection{Take Home Exam}\label{take-home-exam}

\begin{itemize}
\tightlist
\item
  Will be given approximately mid point between Midterm and Final
\item
  Demonstrate ability to obtain results through software and interpret
  findings
\item
  One day to complete and turn in

  \begin{itemize}
  \tightlist
  \item
    Likely will be a Monday with no lab scheduled for that day
  \end{itemize}
\item
  Similar to Homework, but work must be your own
\end{itemize}

\subsection{Expectations and Policies}\label{expectations-and-policies}

\begin{itemize}
\item
  Expecations you can have of me

  \begin{itemize}
  \item
    You should expect me to provide feedback on homeworks and exams in a
    timely fashion
  \item
    You should expect me to be responsive to your questions and
    concerns. If you have emailed me and not received a response with 24
    hours, please feel free to email again. It is best to use my VUMC
    rather than Vanderbilt email address.
  \end{itemize}
\item
  Attendance. The course is offered in-person. If you expect to be
  absent, please let me know and make plans to catch up. Class will
  start on time.
\item
  Collaboration

  \begin{itemize}
  \item
    Discussing course content is highly encouraged
  \item
    Collaborating on homeworks is highly encouraged, but you need to
    turn in your own assignment written in your own words
  \item
    Exams (take home and in class) are individual effort
  \end{itemize}
\item
  Academic honesty

  \begin{itemize}
  \item
    Students are expected to follow the Vanderbilt Honor Code
  \item
    ``Vanderbilt University students pursue all academic endeavors with
    integrity. They conduct themselves honorably, professionally and
    respectfully in all realms of their studies in order to promote and
    secure an atmosphere of dignity and trust. The keystone of our honor
    system is self-regulation, which requires cooperation and support
    from each member of the University community.''
  \end{itemize}
\item
  Use of generative AI algorithms such as ChatGPT

  \begin{itemize}
  \item
    ``To ensure all students have an equal opportunity to succeed and to
    preserve the integrity of the course, students are not permitted to
    submit text that is generated by artificial intelligence (AI)
    systems such as ChatGPT, Bing Chat, Claude, Google Bard, or any
    other automated assistance for any classwork or assessments. This
    includes using AI to generate answers to assignments, exams, or
    projects, or using AI to complete any other course-related tasks.
    Using AI in this way undermines your ability to develop critical
    thinking, writing, or research skills that are essential for this
    course and your academic success. Students may use AI as part of
    their research and preparation for assignments, or as a text editor,
    but text that is submitted must be written by the student. For
    example, students may use AI to generate ideas, questions, or
    summaries that they then revise, expand, or cite properly. AI could
    also be used to assist in writing code as long as the prompts/tools
    are cited. Students should be aware of the potential benefits and
    limitations of using AI as a tool for learning and research. AI
    systems can provide helpful information or suggestions, but they are
    not always reliable or accurate. Students should critically evaluate
    the sources, methods, and outputs of AI systems. Violations of this
    policy will be treated as academic misconduct. If you have any
    questions about this policy or if you are unsure whether a
    particular use of AI is acceptable, please do not hesitate to ask
    for clarification.''
  \item
    Points of emphasis: Crtically evaluate your use of AI and observe
    the above restrictions. When using AI, you \textbf{must}

    \begin{itemize}
    \item
      Cite any text that AI generated (even if you edited it) with a
      bibliographic entry indicating the name and version of the AI
      model that you used, the date and time it was used, and the exact
      query or prompt to generate the results
    \item
      Cite as above any code that was generated for you. I recommend
      that you do not use AI to blindly write code for you. Doing so
      will probably be more work than simply writing the code yourself.
      You must verify any code that is generated for you is accurate and
      answers the question in the assignment instructions.
    \end{itemize}
  \item
    Vanderbilt or VUMC resources

    \begin{itemize}
    \item
      \url{https://www.vanderbilt.edu/generative-ai/}
    \item
      \url{https://www.vumc.org/dbmi/GenerativeAI}
    \end{itemize}
  \end{itemize}
\item
  Late work

  \begin{itemize}
  \item
    While I expect that work will be turned in on time, things can
    happen to interrupt your schedule
  \item
    My goal is to provide sufficient time for completing all assignments
  \item
    If you anticipate a problem with a due date, it is best to let me
    know sooner (e.g.~when a homework is assigned) rather than later
  \item
    Late homeworks will be accepted on a case by case basis. No late
    homeworks will be accepted after the key is provided.
  \end{itemize}
\item
  Voicing concerns and evaluations

  \begin{itemize}
  \item
    Please feel free to bring up any concerns you have about the course
    material, how it is being presented, or how you are being evaluated
    at any time during the semester. I want you to know that your voice
    will be heard.
  \item
    Please complete the end of course evaluations. They are a valuable
    resource for me and help to guide changes from year to year. I read
    all comments and will take them seriously. Comments about what
    worked well as well as constructive criticism are appreciated
  \end{itemize}
\end{itemize}

\subsection{Accommodations}\label{accommodations}

\begin{itemize}
\item
  I encourage students who encounter accessibility challenges to
  communicate with me regardless of whether they are registered with
  Equal Opportunity and Access
\item
  Please communicate with me at your earliest convenience so we can
  discuss specific actions to address your needs. I will make every
  effort to accommodate reasonable requests.
\item
  If you have established accommodations with Equal Opportunity and
  Access, I will receive an email notifying me of the request.
\item
  If you need to contact Equal Opportunity and Access to establish
  service, the address is
  \url{https://www.vanderbilt.edu/eeo/disability_services/contact_us.php}
\end{itemize}




\end{document}
